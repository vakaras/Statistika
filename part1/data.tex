\chapter{Analizuoti duomenys}

Duomenys analizei buvo renkami semestro metu su programa
\emph{Project Hamster}\footnote{
Projekto svetainės adresas: \url{http://projecthamster.wordpress.com/}}
bei eksportuoti iš paskaitų kalendoriaus. Kiekviename stebėjime buvo
fiksuojami tokie kintamieji:
\begin{description}
  \item[name] – normalizuotas dalyko pavadinimas;
  \item[start date] – pradžios data;
  \item[start time] – pradžios laikas;
  \item[stop date] – pabaigos data;
  \item[stop time] – pabaigos laikas;
  \item[type id] – dalyko tipo id.
  \item[category id] – TODO 9 – darbas ne paskaitos metu, 1000 – paskaita.
% TODO Taip pat nepamiršti įtraukti category id į lentelę!
\end{description}

\emph{Pastaba:} Nei vienas stebėjimas netruko ilgiau nei parą.

Analizuotų duomenų fragmentas:

\begin{tabularx}{\textwidth}{p{6em} X X X X X}
  { \bf name }              & { \bf start date} & { \bf start time} & { \bf stop date} 
  &  { \bf stop time} & { \bf type id}\\
  Algoritmavimo seminaras   & 2011-01-22  & 11:39:00   & 2011-01-22 &  12:20:00   & 1\\
  Algoritmavimo seminaras   & 2011-01-22  & 12:36:00   & 2011-01-22 &  14:14:00   & 1\\
  Algoritmavimo seminaras   & 2011-01-22  & 14:46:00   & 2011-01-22 &  16:09:00   & 1\\
  %Algoritmavimo seminaras   & 2011-01-22  & 20:52:00   & 2011-01-22 &  21:06:00   & 1\\
  %Algoritmavimo seminaras   & 2011-01-24  & 09:26:00   & 2011-01-24 &  09:42:00   & 1\\
  %Algoritmavimo seminaras   & 2011-01-24  & 10:17:00   & 2011-01-24 &  11:25:00   & 1\\
  %Algoritmavimo seminaras   & 2011-05-29  & 19:09:00   & 2011-05-29 &  21:18:00   & 1\\
  %Algoritmavimo seminaras   & 2011-05-31  & 21:21:00   & 2011-05-31 &  22:25:00   & 1\\
 %Algoritmavimo seminaras   & 2011-06-02  & 20:08:00   & 2011-06-02 &  20:55:00   & 1\\
 %Algoritmavimo seminaras   & 2011-06-02  & 21:03:00   & 2011-06-02 &  21:29:00   & 1\\
 %Algoritmavimo seminaras   & 2011-06-02  & 21:54:00   & 2011-06-02 &  22:01:00   & 1\\
 %Algoritmavimo seminaras   & 2011-06-02  & 22:18:00   & 2011-06-02 &  23:20:00   & 1\\
 %Algoritmavimo seminaras   & 2011-06-03  & 12:40:00   & 2011-06-03 &  13:06:00   & 1\\
 %Algoritmavimo seminaras   & 2011-06-18  & 09:29:00   & 2011-06-18 &  09:33:00   & 1\\
 Interneto technologijos   & 2011-02-09  & 14:08:00   & 2011-02-09 &  15:36:00   & 2\\
 Interneto technologijos   & 2011-02-09  & 14:00:00   & 2011-02-09 &  16:00:00   & 2\\
 Interneto technologijos   & 2011-02-09  & 12:00:00   & 2011-02-09 &  14:00:00   & 2\\
 Interneto technologijos   & 2011-02-23  & 14:00:00   & 2011-02-23 &  16:00:00   & 2\\
 Interneto technologijos   & 2011-02-23  & 12:00:00   & 2011-02-23 &  14:00:00   & 2\\
 Interneto technologijos   & 2011-03-02  & 14:00:00   & 2011-03-02 &  16:00:00   & 2\\
 %Interneto technologijos   & 2011-03-02  & 12:00:00   & 2011-03-02 &  14:00:00   & 2\\
 %Interneto technologijos   & 2011-03-09  & 14:00:00   & 2011-03-09 &  16:00:00   & 2\\
 %Interneto technologijos   & 2011-03-09  & 12:00:00   & 2011-03-09 &  14:00:00   & 2\\
 %Interneto technologijos   & 2011-03-16  & 14:00:00   & 2011-03-16 &  16:00:00   & 2\\
 %Interneto technologijos   & 2011-03-16  & 12:00:00   & 2011-03-16 &  14:00:00   & 2\\
 %Interneto technologijos   & 2011-03-23  & 14:00:00   & 2011-03-23 &  16:00:00   & 2\\
 %Interneto technologijos   & 2011-03-23  & 12:00:00   & 2011-03-23 &  14:00:00   & 2\\
\end{tabularx}

Pastabos apie studijuotus dalykus:
\begin{enumerate}
  \item
    \label{note:algoritmavimo_seminaras}
    \emph{Algoritmavimo seminaras.} 
    Paskaitų nėra. Įskaita gaunama už atliktas užduotis.
  \item
    \label{note:interneto_technologijos}
    \emph{Interneto technologijos}
    4 valandos paskaitų per savaitę. (2 teorijos, 2 laboratorinių.)
    Pažymys susideda iš gautų laboratorinių darbų ir egzamino 
    įvertinimų.
  \item
    \label{note:matematine_logika}
    \emph{Matematinė logika}
    4 valandos paskaitų per savaitę. (2 teorijos, 2 praktikos.)
    Pažymys susideda iš darbo praktikų metu, kontrolinio ir egzamino
    įvertinimų.
  \item
    \label{note:operacines_sistemos}
    \emph{Operacinės sistemos}
    4 valandos paskaitų per savaitę. (2 teorijos, 2 praktikos.)
    Pažymys gaunamas už praktinės užduoties (realios operacinės
    sistemos sukūrimas) atlikimą.
  \item
    \label{note:programu_sistemu_inzinerija}
    \emph{Programų sistemų inžinerija}
    4 valandos paskaitų per savaitę. (2 teorijos, 2 praktikos.)
    Pažymys susideda iš dviejų laboratorinių darbų ir egzamino įvertinimų.
  \item
    \label{note:psichologijos_ivadas}
    \emph{Psichologijos įvadas}
    2 teorijos paskaitos per savaitę.
    Pažymys susideda iš kontrolinio ir egzamino įvertinimų.
  \item
    \label{note:tikimybiu_teorija}
    \emph{Tikimybių teorija ir matematinė statistika}
    4 valandos paskaitų per savaitę. (2 teorijos, 2 praktikos.)
    Pažymys susideda iš darbo praktikos paskaitų metu, trijų kontrolinių
    ir egzamino įvertinimų.
  \item 
    \label{note:vokieciu_kalba}
    \emph{Vokiečių kalba}
    4 valandos paskaitų per savaitę.
    Įskaita gaunama už namų darbų (jie būdavo užduodami kiekvienai
    paskaitai) bei kontrolinių (kontroliniai būdavo kas 2-4 savaites)
    rezultatus.
\end{enumerate}

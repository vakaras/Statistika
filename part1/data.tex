\chapter{Analizuoti duomenys}

Duomenys analizei buvo renkami semestro metu su programa
\emph{Project Hamster}\footnote{
Projekto svetainės adresas: \url{http://projecthamster.wordpress.com/}}
bei eksportuoti iš paskaitų kalendoriaus. Kiekviename stebėjime buvo
fiksuojami tokie kintamieji:
\begin{description}
  \item[name] – normalizuotas dalyko pavadinimas;
  \item[start date] – pradžios data;
  \item[start time] – pradžios laikas;
  \item[stop date] – pabaigos data;
  \item[stop time] – pabaigos laikas;
  \item[type id] – dalyko tipo id.
  \item[category id] – vienas iš dviejų: 9 – jei darbas ne paskaitos metu, 
    1000 – jei paskaita.
\end{description}
Analizuotų duomenų fragmentas pateiktas \ref{tab:data_sample} lentelėje.

\emph{Pastaba:} Nei vienas stebėjimas netruko ilgiau nei parą.


\begin{sidewaystable}[ht!]
  \centering
  \begin{tabular}{|l|c|c|c|c|c|c|c|}
\hline
  { \bf name }              & { \bf start date} & { \bf start time} & { \bf stop date} 
  &  { \bf stop time} & { \bf type id} & { \bf category id}\\
\hline
Algoritmavimo seminaras&2011-01-22&11:39:00&2011-01-22&12:20:00&1&9\\
Algoritmavimo seminaras&2011-06-18&09:29:00&2011-06-18&09:33:00&1&9\\
Interneto technologijos&2011-02-09&14:08:00&2011-02-09&15:36:00&2&9\\
Interneto technologijos&2011-02-09&14:00:00&2011-02-09&16:00:00&2&1000\\
Interneto technologijos&2011-03-23&14:00:00&2011-03-23&16:00:00&2&1000\\
Interneto technologijos&2011-03-23&12:00:00&2011-03-23&14:00:00&2&1000\\
Interneto technologijos&2011-03-30&13:36:00&2011-03-30&15:49:00&2&9\\
Interneto technologijos&2011-03-30&16:05:00&2011-03-30&17:06:00&2&9\\
Interneto technologijos&2011-03-30&14:00:00&2011-03-30&16:00:00&2&1000\\
\hline
  \end{tabular}
  \caption{Duomenų pavyzdys.}
  \label{tab:data_sample}
\end{sidewaystable}

Pastabos apie studijuotus dalykus:
\begin{enumerate}
  \item
    \label{note:algoritmavimo_seminaras}
    \emph{Algoritmavimo seminaras.} 
    Paskaitų nėra. Įskaita gaunama už atliktas užduotis.
  \item
    \label{note:interneto_technologijos}
    \emph{Interneto technologijos}
    4 valandos paskaitų per savaitę. (2 teorijos, 2 laboratorinių.)
    Pažymys susideda iš gautų laboratorinių darbų ir egzamino 
    įvertinimų.
  \item
    \label{note:matematine_logika}
    \emph{Matematinė logika}
    4 valandos paskaitų per savaitę. (2 teorijos, 2 praktikos.)
    Pažymys susideda iš darbo praktikų metu, kontrolinio ir egzamino
    įvertinimų.
  \item
    \label{note:operacines_sistemos}
    \emph{Operacinės sistemos}
    4 valandos paskaitų per savaitę. (2 teorijos, 2 praktikos.)
    Pažymys gaunamas už praktinės užduoties (realios operacinės
    sistemos sukūrimas) atlikimą.
  \item
    \label{note:programu_sistemu_inzinerija}
    \emph{Programų sistemų inžinerija}
    4 valandos paskaitų per savaitę. (2 teorijos, 2 praktikos.)
    Pažymys susideda iš dviejų laboratorinių darbų ir egzamino įvertinimų.
  \item
    \label{note:psichologijos_ivadas}
    \emph{Psichologijos įvadas}
    2 teorijos paskaitos per savaitę.
    Pažymys susideda iš kontrolinio ir egzamino įvertinimų.
  \item
    \label{note:tikimybiu_teorija}
    \emph{Tikimybių teorija ir matematinė statistika}
    4 valandos paskaitų per savaitę. (2 teorijos, 2 praktikos.)
    Pažymys susideda iš darbo praktikos paskaitų metu, trijų kontrolinių
    ir egzamino įvertinimų.
  \item 
    \label{note:vokieciu_kalba}
    \emph{Vokiečių kalba}
    4 valandos paskaitų per savaitę.
    Įskaita gaunama už namų darbų (jie būdavo užduodami kiekvienai
    paskaitai) bei kontrolinių (kontroliniai būdavo kas 2-4 savaites)
    rezultatus.
\end{enumerate}

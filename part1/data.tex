\chapter{Analizuoti duomenys}

Duomenys analizei buvo renkami semestro metu su programa \emph{Project 
Hamster}\footnote{
Projekto svetainės adresas: \url{http://projecthamster.wordpress.com/}}
bei eksportuoti iš paskaitų kalendoriaus.

Kiekviename įraše turimi tokie parametrai:
\begin{description}
  \item[name] – normalizuotas dalyko pavadinimas;
  \item[start date] – pradžios data;
  \item[start time] – pradžios laikas;
  \item[stop date] – pabaigos data;
  \item[stop time] – pabaigos laikas;
  \item[type id] – dalyko tipas id.
\end{description}

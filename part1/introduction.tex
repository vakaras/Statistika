\chapter{Įvadas}

Šis dokumentas yra dalyko „Statistinė duomenų analizė“ laboratorinių darbų
savarankiška užduotis.

\section{Darbo tikslas}

Darbo tikslas 

\section{Temos aktualumas}

TODO

\section{Siekiami rezultatai}

TODO


\section{Analizuoti duomenys}

\begin{description}
  \item[data] stebėjimo atlikimo data;
  \item[kokybinis] veiklos tipas;
  \item[kiekybinis tolydus] veiklos trukmė;
  \item[kiekybinis diskretus] kiek kartų buvo vykdyta veikla.
\end{description}

Analizuojamos trukmės charakteristikos:
\begin{description}
  \item[n] – stebėjimų panaudotų skaičiavimuose, skaičius;
  \item[sum] – stebėjimų suma;
  \item[p25] – pirmas kvartilis;
  \item[p50] – mediana;
  \item[p75] – trečias kvartilis;
  \item[mean] – aritmetinis vidurkis;
  \item[max] – didžiausia reikšmė;
  \item[min] – mažiausia reikšmė;
  \item[std] – standartinis nuokrypis;
  \item[skew] – asimetrijos koeficientas (44 psl.);
  \item[kurt] – eksceso koeficientas (44 psl.);
\end{description}
